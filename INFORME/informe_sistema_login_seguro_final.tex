\documentclass[12pt,a4paper]{article}
\usepackage[spanish]{babel}
\usepackage{graphicx}
\usepackage{geometry}
\usepackage{fancyhdr}
\usepackage{setspace}
\usepackage{hyperref}
\usepackage{color}
\usepackage{listings}
\usepackage{xcolor}
\usepackage{float}
\usepackage{amsmath}
\usepackage{amssymb}
\usepackage{booktabs}
\usepackage{multirow}
\usepackage{helvet}

% Usar Arial como fuente principal
\renewcommand{\familydefault}{\sfdefault}

% Ajuste de headheight
\setlength{\headheight}{15pt}

% Configuración de márgenes - A4
\geometry{
  paper=a4paper,
  top=2.5cm,
  bottom=2.5cm,
  left=2.5cm,
  right=2.5cm
}

% Configuración de espaciado reducido
\onehalfspacing

% Reducir espaciado entre items
\usepackage{enumitem}
\setlist[description]{itemsep=0pt, parsep=0pt, leftmargin=2cm}
\setlist[enumerate]{itemsep=2pt, parsep=2pt}
\setlist[itemize]{itemsep=2pt, parsep=2pt}

% Configuración de encabezado y pie
\pagestyle{fancy}
\fancyhf{}
\fancyhead[L]{\textit{Sistema de Login Seguro}}
\fancyhead[R]{\thepage}
\fancyfoot[C]{\textit{Criptografía y Ciberseguridad - TP Final}}
\renewcommand{\headrulewidth}{0.5pt}
\renewcommand{\footrulewidth}{0.5pt}

% Configuración de código
\lstset{
  basicstyle=\ttfamily\small,
  keywordstyle=\color{blue},
  commentstyle=\color{gray},
  stringstyle=\color{red},
  breaklines=true,
  numbers=left,
  numberstyle=\tiny,
  backgroundcolor=\color{lightgray!30},
  frame=single,
  captionpos=b
}

% Información del documento
\title{
  \vspace{2cm}
  \includegraphics[width=4cm]{image.png}\\[1cm]
  {\LARGE \textbf{SISTEMA DE AUTENTICACIÓN SEGURO}}\\
  {\large Diseño e Implementación de Mejores Prácticas en Criptografía}\\
  \vspace{0.5cm}
  {\normalsize Trabajo Práctico Final - Noviembre 2025}
}

\author{\textbf{Criptografía y Ciberseguridad}}
\date{}

\begin{document}

\maketitle

\vspace{0.5cm}

\section*{RESUMEN EJECUTIVO}

El presente informe detalla el diseño de un sistema de autenticación seguro. Se utilizan algoritmos de hashing modernos (Argon2id), tokens JWT con rotación automática, refresh token stateful, y múltiples controles preventivos. El sistema cumple con OWASP Top 10, NIST SP 800-63B y principios de privacidad por diseño.

\section{REQUISITOS FUNCIONALES Y NO FUNCIONALES}

\subsection{Requisitos Funcionales (RF)}

Los RF especifican qué funcionalidades debe proporcionar el sistema:

\begin{description}
  \item[RF1 - Registro] El sistema permite crear una cuenta con email y contraseña únicos.
  \item[RF2 - Verificación de Email] Tras el registro, el usuario verifica su email mediante enlace único y de corta duración.
  \item[RF3 - Login] El usuario autenticado con email y contraseña correctos recibe tokens de sesión.
  \item[RF4 - Logout] El usuario cierra su sesión de forma segura revocando todos los tokens.
  \item[RF5 - Cambio de Contraseña] El usuario autenticado cambia su contraseña (requiriendo la antigua).
  \item[RF6 - Recuperación de Contraseña] Usuario que olvidó su contraseña recibe enlace de reseteo por email.
  \item[RF7 - Gestión de Sesión (Refresh)] El sistema renueva sesiones expiradas sin pedir credenciales, mediante refresh token.
\end{description}

\subsection{Requisitos No Funcionales (RNF)}

Los RNF especifican cómo el sistema implementa seguridad y calidad:

\begin{description}
  \item[RNF1 - Confidencialidad] Las contraseñas se almacenan hasheadas (Argon2id), nunca en texto plano.
  \item[RNF2 - Resistencia a Fuerza Bruta] Rate limiting en endpoints críticos (login, recuperación).
  \item[RNF3 - Transporte Seguro] Comunicación cifrada con HTTPS/TLS 1.2+.
  \item[RNF4 - Gestión Segura de Sesión] Sesiones gestionadas por JWT (acceso) + cookies seguras (refresh).
  \item[RNF5 - Prevención XSS] Tokens de sesión en cookies HttpOnly, no accesibles por JavaScript.
  \item[RNF6 - Prevención CSRF] Cookies con flag SameSite=Strict.
  \item[RNF7 - Mínimo Privilegio] Tokens contienen solo user\_id y roles, sin información sensible (PII).
\end{description}

\section{MODELO DE DATOS}

El modelo prioriza seguridad utilizando UUIDs como claves primarias para mitigar enumeración de recursos.

\subsection{Tabla: Usuarios}

\begin{table}[H]
  \centering
  \small
  \begin{tabular}{|l|l|p{3cm}|}
    \hline
    \textbf{Columna} & \textbf{Tipo} & \textbf{Descripción} \\
    \hline
    id & UUID PK & Identificador único (UUID v4) \\
    \hline
    email & VARCHAR UNIQUE & Email como identificador de login \\
    \hline
    password\_hash & VARCHAR & Hash Argon2id de la contraseña \\
    \hline
    email\_verified & BOOLEAN & Flag de verificación de email \\
    \hline
    created\_at & TIMESTAMP & Fecha de creación \\
    \hline
    updated\_at & TIMESTAMP & Fecha de última modificación \\
    \hline
  \end{tabular}
\end{table}

\subsection{Tabla: Sesiones}

Tabla fundamental para gestión stateful de refresh tokens, permitiendo revocación y rotación.

\begin{table}[H]
  \centering
  \small
  \begin{tabular}{|l|l|p{2.5cm}|}
    \hline
    \textbf{Columna} & \textbf{Tipo} & \textbf{Descripción} \\
    \hline
    id & UUID PK & Identificador de sesión \\
    \hline
    user\_id & UUID FK & Usuario propietario \\
    \hline
    refresh\_token\_hash & VARCHAR UNIQUE & Hash SHA-256 del refresh token \\
    \hline
    ip\_address & VARCHAR & IP de origen (auditoría) \\
    \hline
    user\_agent & TEXT & Metadata del cliente \\
    \hline
    expires\_at & TIMESTAMP & Expiración del refresh token \\
    \hline
    created\_at & TIMESTAMP & Creación de sesión \\
    \hline
  \end{tabular}
\end{table}

\textbf{Nota Crítica:} El hash del refresh token se almacena, no el token en texto plano. Si la BD fuese comprometida, los atacantes no podrían usar los tokens directamente.

\section{FLUJOS DE AUTENTICACIÓN}

\subsection{Flujo de Login y Emisión de Tokens}

\begin{enumerate}
  \item \textbf{Petición}: Cliente envía email y password a \texttt{/api/login}.
  \item \textbf{Rate Limiting}: Servidor aplica límites para mitigar fuerza bruta.
  \item \textbf{Verificación}: Se busca el usuario por email y verifica contraseña con \texttt{argon2.verify()}. Error genérico 401 para evitar enumeración.
  \item \textbf{Generación de Tokens} (exitosa):
    \begin{itemize}
      \item \textbf{Access Token (JWT)}: Payload mínimo (sub: user.id), expiración 15 min, algoritmo RS256/ES256.
      \item \textbf{Refresh Token (Opaque)}: String aleatorio 32 bytes, expiración 7 días.
    \end{itemize}
  \item \textbf{Persistencia}: Hash SHA-256 del refresh token se almacena en tabla Sesiones.
  \item \textbf{Respuesta}: Ambos tokens enviados en cookies HttpOnly+Secure+SameSite=Strict.
\end{enumerate}

\subsection{Flujo de Refresh y Rotación de Tokens}

\begin{enumerate}
  \item \textbf{Contexto}: Access token expira, cliente detecta 401 en petición protegida.
  \item \textbf{Petición}: Cliente envía refresh token a \texttt{/api/refresh}.
  \item \textbf{Validación}:
    \begin{itemize}
      \item Servidor hashea el refresh token recibido.
      \item Busca hash en tabla Sesiones.
      \item Si no existe o expiró: 401 Unauthorized, re-login requerido.
    \end{itemize}
  \item \textbf{Rotación} (implementación clave):
    \begin{itemize}
      \item Token válido se \textbf{elimina inmediatamente} de BD (invalidando su uso futuro).
      \item Se genera nuevo access token (15 min).
      \item Se genera nuevo refresh token (7 días).
      \item Hash del nuevo refresh token se almacena en Sesiones.
    \end{itemize}
  \item \textbf{Justificación}: Si el token es comprometido, el usuario legítimo será desautenticado en su próximo refresh (token ya invalidado por atacante). Detecta y contiene brechas.
  \item \textbf{Respuesta}: Nuevos tokens en cookies seguras.
\end{enumerate}

\subsection{Flujo de Logout (Revocación)}

\begin{enumerate}
  \item \textbf{Petición}: Cliente envía petición a \texttt{/api/logout} (con refresh token).
  \item \textbf{Revocación}: Servidor hashea token, busca y \textbf{elimina} registro en Sesiones.
  \item \textbf{Respuesta}: 204 No Content.
  \item \textbf{Limpieza Cliente}: Aplicación borra tokens de su almacenamiento.
\end{enumerate}

Este mecanismo stateful permite revocación permanente e inmediata del lado del servidor, a diferencia de JWTs de acceso (stateless).

\section{DIAGRAMAS DE SECUENCIA}

Los siguientes diagramas ilustran los flujos críticos del sistema de autenticación.

\begin{figure}[H]
  \centering
  \includegraphics[width=0.9\textwidth]{1.png}
  \caption{Flujo de Registro: El cliente envía email y contraseña. El servidor valida, hashea la contraseña con Argon2id, y almacena. Se genera token de verificación de email y se envía asincronamente.}
  \label{fig:registro}
\end{figure}

\begin{figure}[H]
  \centering
  \includegraphics[width=0.9\textwidth]{2.png}
  \caption{Flujo de Login: Después de validaciones y rate limiting, se generan access token (JWT, 15 min) y refresh token (opaco, 7 días). Ambos se envían en cookies seguras.}
  \label{fig:login}
\end{figure}

\begin{figure}[H]
  \centering
  \includegraphics[width=0.9\textwidth]{4.png}
  \caption{Flujo de Logout: El servidor revoca el refresh token eliminando su hash de la tabla Sesiones. El acceso token se añade a lista negra. Cliente limpia almacenamiento.}
  \label{fig:logout}
\end{figure}

\begin{figure}[H]
  \centering
  \includegraphics[width=0.9\textwidth]{3.png}
  \caption{Flujo de Refresh: Rotación de tokens - token antiguo se elimina inmediatamente. Si se detecta reuso, se revocan todos los tokens de la familia (indicador de compromiso).}
  \label{fig:refresh}
\end{figure}

\section{DECISIONES CRIPTOGRÁFICAS}

\subsection{Hashing de Contraseñas: Argon2id}

\textbf{Algoritmo}: Argon2id es la recomendación actual de OWASP y ganador de Password Hashing Competition (2015).

\textbf{Ventajas}:
\begin{itemize}
  \item \textbf{Memory-Hard}: Incrementa costo de fuerza bruta tanto en tiempo como en memoria.
  \item \textbf{GPU/ASIC Resistance}: Resistencia superior al hardware especializado vs. bcrypt/PBKDF2.
  \item \textbf{Moderna}: Diseñada considerando tendencias de hardware actual.
\end{itemize}

\textbf{Parámetros OWASP}: memoryCost=19456, timeCost=2, parallelism=1.

\subsection{Firmas de Tokens: RS256 o ES256}

\textbf{Selección}: Algoritmos asimétricos (clave pública/privada) en lugar de HS256.

\textbf{Ventajas}:
\begin{itemize}
  \item Servicio de autenticación firma con clave privada.
  \item Microservicios verifican con clave pública.
  \item Evita compartir secreto simétrico entre servicios.
  \item Escalabilidad y seguridad mejoradas en arquitecturas distribuidas.
\end{itemize}

\section{GESTIÓN DE SESIONES Y ALMACENAMIENTO DE TOKENS}

\subsection{Método: Cookies con Seguridad Reforzada}

Se descarta LocalStorage/SessionStorage por vulnerabilidad inherente a XSS. Los tokens se almacenan en cookies con atributos de seguridad:

\begin{itemize}
  \item \textbf{HttpOnly=True}: Previene acceso desde JavaScript, mitiga XSS.
  \item \textbf{Secure=True}: Solo transmitidas por HTTPS, previene man-in-the-middle.
  \item \textbf{SameSite=Strict}: Restringe envío a peticiones del mismo origen, previene CSRF.
  \item \textbf{Path Específico}: Refresh token con Path=/api/refresh limita scope.
\end{itemize}

\subsection{Inferencia de Estado en Frontend}

Como tokens no son accesibles por JavaScript, el cliente infiere estado de autenticación realizando petición a endpoint protegido (ej. \texttt{/api/me}):
\begin{itemize}
  \item \textbf{200 OK}: Sesión válida, datos del usuario retornados.
  \item \textbf{401 Unauthorized}: Usuario no autenticado.
\end{itemize}

\section{PRIVACIDAD Y MINIMIZACIÓN DE DATOS}

\subsection{Minimización de Datos}

\textbf{Registro}: Solo información estrictamente necesaria para autenticación (email, contraseña).

\textbf{JWT Payload}: Access tokens contienen solo user\_id (UUID) y roles, excluyendo PII. Si endpoint requiere datos del usuario, consulta BD usando user\_id del token.

\subsection{Políticas de Retención}

\begin{itemize}
  \item \textbf{Sesiones}: Cron job purga registros con expires\_at vencido.
  \item \textbf{Logs}: Anonimizados o eliminados tras 90 días de retención.
\end{itemize}

\subsection{Derecho al Olvido}

Sistema provee endpoint de eliminación de cuenta (previa re-autenticación):
\begin{itemize}
  \item Hard delete del registro en tabla Usuarios.
  \item ON DELETE CASCADE elimina en cascada registros en Sesiones.
\end{itemize}

\section{CONCLUSIONES}

El sistema de autenticación diseñado implementa múltiples capas de seguridad: hashing moderno (Argon2id), tokens de corta duración con rotación automática, refresh tokens stateful para revocación, cookies seguras contra XSS/CSRF, y privacidad por diseño. Cumple con estándares OWASP, NIST SP 800-63B y principios de minimización de datos.

Las mejoras futuras incluyen autenticación multifactor (TOTP, WebAuthn), análisis de riesgo adaptativo, y arquitectura zero trust. La seguridad es un proceso continuo que requiere revisiones de código, pentesting, y actualizaciones de dependencias.

\newpage

\section*{REFERENCIAS}

\begin{enumerate}
  \item OWASP. (2021). OWASP Top Ten 2021. https://owasp.org/Top10/
  \item OWASP. (2023). Authentication Cheat Sheet. https://cheatsheetseries.owasp.org/
  \item NIST. (2020). SP 800-63B: Digital Identity Guidelines - Authentication.
  \item RFC 7519: JSON Web Token (JWT). https://tools.ietf.org/html/rfc7519
  \item Biryukov, A., Dinu, D., \& Khovratovich, D. (2016). Argon2: Memory-hard function for password hashing.
  \item OWASP. Password Storage Cheat Sheet. https://cheatsheetseries.owasp.org/
  \item Mozilla. Web Security Guidelines. https://infosec.mozilla.org/guidelines/web\_security
  \item EU GDPR (2016/679). Reglamento General de Protección de Datos.
\end{enumerate}

\end{document}
